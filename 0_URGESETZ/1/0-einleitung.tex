\section{Einleitung}

Dieses Buch will dem kranken suchenden Menschen helfen, den
Weg zur ersehnten vollkommenen Gesundheit zu finden. Dieser ist
ihm in der naturgegebenen einfachen Ernährung von dem, was in
jedem Garten wächst und angebaut werden kann, von Anbeginn an
mit auf den Lebensweg gegeben, aber zum Unglück für den Menschen
nicht eingehalten worden. Der Ackerbau treibende Kulturmensch
unserer gepriesenen Zivilisation lebt vom Ertrag des Ackerbaues,
d. h. von Fleisch, Milch und Brot neben gekochten Feldgemüsen. Die
Menschheit in ihrem augenblicklichen Kulturzustand lebt und ernährt
sich nach dem Grundsatz:

Wald - Axt - Kuh - Acker - Wüste

Zur Erläuterung dieser fünf Worte diene folgendes: Um Fleisch und
vom Tier stammende Genußmittel essen und Milch trinken zu können,
griff der Mensch einst zur Axt und schlug die schützende Walddecke
der Erde mitsamt seinen eigenen Gartenanlagen nieder, um Rinder,
Schafe, Pferde und im Orient Kamele usw. ernähren zu können. Ge­
nügte die dadurch gewonnene Viehweide nicht, um die wachsende
Bevölkerung mit Fleisch, Milch und Brot sättigen zu können, dann
griff der Mensch zum Pflug, riß den Boden auf und betrieb fortan
Ackerbau zur Erzeugung von Viehfutter. Dabei entdeckte er vor Zei­
ten durch den Anbau von Feldfrüchten, Futterrüben und Gräsern
zur Heugewinnung für den Winter, die Möglichkeit der Ernährung
aus den Körnern hochgezüchteter Gräser auf dem Umweg über das
weichmachende Feuer. Sie schufen sich das gebackene Brot. Fleisch,
Milch und Brot wurden die Hauptnahrung der Menschen. Diese Art
der Sättigung durch Brot, Milch und Fleisch von Tierleichen aber ist
nicht naturgegeben. Die Organe des menschlichen Körpers sind von
Natur aus auf die Verarbeitung und Umwandlung von Gartenge­
müsen, Obst und Nüssen im Naturzustand eingerichtet und erschaffen.
Durch das Verspeisen von Brot, Milch und Fleisch samt dem gekochten
Grobgemüse müssen die Organe entarten. Dieser Zustand zeigt sich
dann in den verschiedensten Krankheitserscheinungen.

Der Boden, die Erde aber wird durch die fortschreitende Entwal­
dung, dort wo der Fleisch, Milch und Brot verzehrende Mensch den
Acker dauernd aufriß und neubestellte, das Grundwasser, das Blut
der Erde, verlieren. Die von Wald und Baumwuchs ungeschützte Erde muß unter den Strahlen der Sonne verdorren und im Laufe der Zeit zur Wüste werden. Die fruchtbare Ackererde wird in trockenen Jah­
ren entweder davonfliegen oder wegen Wassermangel versteppen und
zuletzt verdorren. Das ist bisher das Schicksal aller Kulturlandschaf­
ten gewesen, deren Bewohner sich von Ackerbau und Viehzucht zu
nähren suchten. Auch die Sahara ist durch Abholzung des einst dort
wachsenden wasserreichen Urwaldes durch menschliche Unvernunft
zur Wüste geworden.

In diesem Buche aber wird dem Menschen die natürliche ihm von
Anbeginn zugewiesene Ernährung gezeigt. Diese ist nach den Ergeb­
nissen der besten biologischen Forschungen mit allen wissenschaft­
lichen Hilfsmitteln und nach langjährigen praktischen Erfahrungen
die einzig richtige, die für den Menschen in Frage kommen kann.
Der Inhalt dieses Buches will dem Leser nicht nur die zum Ver­
ständnis notwendigen wissenschaftlichen Erkenntnisse vermitteln,
sondern ihm ganz besonders eindringlich die praktischen Erfahrungen
bringen. Nach diesen kann er sich richten, um seinen Körper in
einen Zustand vollkommener Gesundheit zu bringen. Die im folgen­
den gezeigte Art der Ernährung beruht auf den natürlichen Gesetzen
der Lebenserhaltung. Sie wird den Menschen, der sie einhalten will,
zu ungeahntem Fortschritt auf allen Gebieten seiner körperlichen,
geistigen und seelischen Fähigkeiten führen. Das aber ist die Vor­
bedingung zur Entwicklung der in den Menschen hineingeborenen
Seelenkräfte in dem gesunden Streben seines Geistes nach Voll­
kommenheit.

Die dadurch ermöglichte Gleichrichtung seiner Willensbestrebungen
mit denen der Natur wird dem Menschen den Frieden seiner Seele wie­
derfinden lassen. Dadurch wird er seine Arbeit und seine Unterneh­
mungen im Sinne der natürlichen Gegebenheiten zu gutem, friedlichem
Ende führen. Jeder folgerichtig Denkende wird durch das Lesen
dieses Buches erkennen lernen, daß alles nur erreicht werden kann
durch einfache Ernährung von natürlich gewachsener pflanzlicher
Nahrung in frischem, lebensvollem Zustand. Diese Nahrung ist nicht
nur wirtschaftlich bekömmlich und einladend, sondern zugleich wohl­
schmeckend, sättigend ·und erfrischend. Sie enthält alle Grundstoffe
für den Aufbau und den Betrieb eines gesunden Körpers.

Es muß hier gleich darauf aufmerksam gemacht werden, daß ge­
kochte Nahrung irgendeiner Art niemals natürlich sein kann, da ihre
Bestandteile durch die zerstörende Kraft der Hitzeeinwirkung bei der
Zubereitung aus ihrer lebenskräftigen, organischen Bindung heraus­
gerissen werden und dadurch für die Erhaltung der Lebenskraft des
Körpers verloren gehen. Durch das Kochen wird die Lebenskraft der
Pflanze vernichtet, die im Pflanzenwuchs gebundene Sonnenkraft und
deren Lichteinwirkung aufgelöst. Die Hitzeeinwirkung löst die auf­
bauenden mineralischen Grundstoffe der Erde aus ihrer organisch
gewachsenen Bindung und bewirkt die Bildung fester, während der
Verdauungstätigkeit nicht mehr zu lösender Verbindungen, die dann
entweder als Ballast oder als Reizgifte im Körper wirken. Schon
Temperaturerhöhungen, die über 43 Grad liegen, bringen die lebens­
spendenden Proteine (Eiweißgebilde) zum Gerinnen und töten da­
durch deren Lebenskraft. Die Stärkekörperchen im Getreide und in
den Wurzeln oder Wurzelknollen werden durch das Koch- oder Back­
verfahren gesprengt und dadurch in Kleister verwandelt. Der ent­
stehende wasserlösliche Kleister verdirbt, als Brot oder Getreide­
speise gegessen, die Verdauungsvorgänge und stört die Wandlung der
Säfte in den feinsten Haargefäßen der Blutbahnen. Er verhindert das
schnelle und störungsfreie Arbeiten der Wandlungsvorgänge in den
feinsten Muskelgewebezellen und ruft dadurch viele krankhafte
Stoffwechselstörungen mit üblen Begleiterscheinungen wie z. B. die
Zuckerharnruhr, hervor. Es ist besser, Brot und gekochte Getreide­
speisen ganz zu meiden, als sich dauernd der Gefahr des Ausbruchs
der verschiedensten Krankheitserscheinungen auszusetzen oder seinen
Körper im späteren Lebensalter verfallen zu sehen. Die organische
Zusammensetzung des natürlichen Frucht-, Trauben- und Wurzel­
zuckers in den natürlich gewachsenen Nahrungsmitteln wird in der
Hitze des Kochens vernichtet. Der Zucker wird dadurch fest und
nicht mehr so leicht wandlungsfähig wie in seinem natürlichen,
organisch gewachsenen Aufbau. Er wird deshalb für den Lebens­
betrieb so gut wie unbrauchbar. Durch den in der Siedehitze erzeug­
ten chemisch reinen Fabrikzucker entstehen Magen- und Darm­
katarrhe verschiedenster Art deshalb, weil der Kunstzucker ein
chemisch reines Erzeugnis ist. Ihm fehlt jeder natürliche Mineral­
stoffgehalt vollständig. Er wirkt deshalb wie eine fressende Säure.
In der Bratpfanne geschmolzene Öle und Fette sind, wie später
bewiesen wird, so gut wie unverdaulich und bewirken daher in den
Lebensvorgängen des Körpers schwere Störungen. Mit anderen Wor­
ten: In der Koch- und Siedehitze veränderte Nahrung wird in den
Verdauungsvorgängen nicht richtig und natürlich verarbeitet, sondern
beginnt dort zu faulen und in Gärung überzugehen. Außerdem
nimmt der weichgekochte Brei den Zähnen die Arbeit und gibt des­
halb auch keine Anregung für den Speichelfluß und die Absonderung
der Magensäfte. Er nimmt den Organen der Verdauung und Um­
wandlung die notwendige Betätigungsmöglichkeit, verwirrt, überreizt
und verdirbt den Säftefiuß der Verdauungsvorgänge und legt damit
die Grundlage zu allen Krankheitserscheinungen.