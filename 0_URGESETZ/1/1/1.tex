\chapter{Was ist Nahrung?}
Die einfachste Antwort auf diese Frage wäre wohl: Was wir essen,
ist unsere Nahrung. Aber der einfache Augenschein belehrt uns, wie
falsch diese Auffassung ist. Wir wissen z. B., daß ein Stück Vieh,
richtig und gesund ernährt, sich prächtig entwickelt, ein glattes Fell
zeigt und gesunde, kräftige Nachkommen zeugen kann. Wir können
demselben Tier etwas als Nahrung verabreichen, durch das es wohl
dick und fett wird, aber gleichzeitig das Fell krankhaft entartet, weil
durch sie die Drüsentätigkeit der inneren Organe und die Blutbildung
gestört wird. Daraus entstehen dann z. B. Knochenmißbildungen, die
bei Mastschweinen oft so weit getrieben werden, daß die Beine den
Körper nicht mehr tragen können. In dem einen Fall sehen wir: Was
wir dem Tier als Futter gaben, verhalf seinem Körper, seinen Organen,
seinem Fleisch, seinen Knochen und seinen Blutgefäßen usw. zur richtigen, gesunden Entwicklung. Im zweiten Beispiel wurde das Tier
durch das, was ihm gereicht wurde, krank. Wirklich gesunde Nahrung
für das Tier ist daher immer nur, was den inneren Aufbau der Organe,
der Haut, der Knochen, der Muskeln und der Gewebe richtig weiter
entwickelt, kräftigt und in bester Gesundheit erhält. Für den Menschen
die Frage „Was ist Nahrung?“ zu beantworten und die Grundgesetze
aufzustellen über das, was für ihn die richtigen und natürlichen
Nahrungsmittel sind, ist die Aufgabe dieses Buches.

Die Beantwortung ist keineswegs so einfach, wie es im ersten Augenblick
aussieht; denn wenn das, was die Menschen heute essen, für ihre
Organe, ihr Blut und ihre Knochen, für die Entwicklung ihrer inneren
Säfte usw. das Richtige wäre, dann müßten sich alle Menschen in
prächtiger Gesundheit entwickeln. Sie müßten alle, ohne Ausnahme, ein
hohes und zufriedenes Alter erreichen und bis zum Schluß schaffensfreudig,
kräftig und leistungsfähig sein. Sie würden dann nicht, wie es
heute an der Tagesordnung ist, vorzeitig und mit Schmerzen sterben,
sondern einem normalen, zufriedenen Alterstod erliegen. Sie würden
dann nicht einem Versagen der inneren Organe zum Opfer fallen oder
an Krankheiten körperlicher, geistiger oder seelischer Art leiden, durch
die sie massenweise in Krankenhäusern, Irrenanstalten und Gefängnissen verderben.
Was ist denn die natürliche Nahrung des Menschen?
Ehe wir diese Frage beantworten können, müssen wir uns ein wenig
in der Natur umsehen, um zu erfassen, aus was eigentlich das Leben
besteht, das wir durch unsere Nahrung aufrecht erhalten wollen.