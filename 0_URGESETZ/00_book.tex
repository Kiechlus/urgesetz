%
% Body text font is Palatino!
%

\documentclass[
a5paper,
pagesize,
10pt,
bibtotoc,
pointlessnumbers,
%normalheadings,
headings=small,
notitlepage,
DIV=16,
twoside]{scrbook}

%\documentclass[a5paper,pagesize,10pt,bibtotoc,pointlessnumbers,
%normalheadings,twoside=true]{scrbook}

% twoside, openright
\KOMAoptions{DIV=last}





\makeatletter
\renewcommand*\l@part[2]{%
   \ifnum \c@tocdepth >-2\relax
     \addpenalty{-\@highpenalty}%
     \addvspace{2.25em \@plus\p@}%
     \setlength\@tempdima{3em}%
     \begingroup
       \parindent \z@ \rightskip \@pnumwidth
       \parfillskip -\@pnumwidth
       {\leavevmode
        \hspace*{\fill}\centering\large\bfseries #1\hspace*{\fill}\llap{#2}}\par
        \nobreak
          \global\@nobreaktrue
          \everypar{\global\@nobreakfalse\everypar{}}%
     \endgroup
   \fi}
\makeatother




\usepackage{trajan}
 
\usepackage[ngerman]{babel}
\usepackage[utf8]{inputenc}
\usepackage[T1]{fontenc}
\usepackage[babel,german=guillemets]{csquotes}

\usepackage[sc]{mathpazo}
%\linespread{1.05} 
\linespread{1} 
\usepackage{verbatim} % for comments
\usepackage{listings} % for comments

%\setlength{\parindent}{10pt}
%\setlength{\parskip}{1.4ex plus 0.35ex minus 0.3ex}
%\setlength{\parskip}{1.4ex plus 0.35ex minus 0.3ex}

\usepackage{blindtext}
\newcommand{\q}[1]{>>\textit{#1}<<}

\title{A book title}   
\author{Author Name} 
\date{\today} 





\begin{document}

%=========================================
\begin{comment}

\begin{titlepage}
		\centering{
			{\fontsize{40}{48}\selectfont 
			A book title}
		}\\
			
		\vspace{10mm}
		\centering{\Large{Author Name}}\\
		\vspace{\fill}
		\centering \large{2011}
\end{titlepage}
\newpage{}
\thispagestyle {empty}

\vspace*{2cm}

\begin{center}
	\Large{\parbox{10cm}{
		\begin{raggedright}
		{\Large 
			\textit{Do what you think is interesting, 
			do something that you think is fun and worthwhile, 
			because otherwise you won’t do it well anyway.}
		}
	
		\vspace{.5cm}\hfill{---Brian W. Kernighan}
		\end{raggedright}
	}
}
\end{center}
\end{comment}
%=========================================

\renewcommand\partname{Buch}
\renewcommand\contentsname{\centering Inhaltsverzeichnis}
\tableofcontents
\newpage
\pagestyle{plain}
\setcounter{secnumdepth}{0} % sections are level 1
\chapter*{Vorwort}
\addcontentsline{toc}{section}{Vorwort}

Aus der Not der Zeit geboren, in der die verängstigte Menschheit
nicht ein und aus weiß vor den Gefahren, die drohende Kriege herauf
beschWör€n, übergebe ich im folgenden dieses Buch der Öffentlichkeit. Es
greift mit unbarmherziger Offenheit die Grundlagen der landesüblichen,
Jahrtausende alten Ernährungsgrundlagen der Menschheit an. Es weist
mit absoluter Wahrhafiigkeit und unbeugsamer Gewissenhafisigkeit nach,
daß alles, was die Menschen bisher als Nahrungs- oder Lebensmittel
\begin{center}\part{Unsere Nahrung}\end{center}
\section{Einleitung}

Dieses Buch will dem kranken suchenden Menschen helfen, den
Weg zur ersehnten vollkommenen Gesundheit zu finden. Dieser ist
ihm in der naturgegebenen einfachen Ernährung von dem, was in
jedem Garten wächst und angebaut werden kann, von Anbeginn an
mit auf den Lebensweg gegeben, aber zum Unglück für den Menschen
nicht eingehalten worden. Der Ackerbau treibende Kulturmensch
unserer gepriesenen Zivilisation lebt vom Ertrag des Ackerbaues,
d. h. von Fleisch, Milch und Brot neben gekochten Feldgemüsen. Die
Menschheit in ihrem augenblicklichen Kulturzustand lebt und ernährt
sich nach dem Grundsatz:

Wald - Axt - Kuh - Acker - Wüste

Zur Erläuterung dieser fünf Worte diene folgendes: Um Fleisch und
vom Tier stammende Genußmittel essen und Milch trinken zu können,
griff der Mensch einst zur Axt und schlug die schützende Walddecke
der Erde mitsamt seinen eigenen Gartenanlagen nieder, um Rinder,
Schafe, Pferde und im Orient Kamele usw. ernähren zu können. Ge­
nügte die dadurch gewonnene Viehweide nicht, um die wachsende
Bevölkerung mit Fleisch, Milch und Brot sättigen zu können, dann
griff der Mensch zum Pflug, riß den Boden auf und betrieb fortan
Ackerbau zur Erzeugung von Viehfutter. Dabei entdeckte er vor Zei­
ten durch den Anbau von Feldfrüchten, Futterrüben und Gräsern
zur Heugewinnung für den Winter, die Möglichkeit der Ernährung
aus den Körnern hochgezüchteter Gräser auf dem Umweg über das
weichmachende Feuer. Sie schufen sich das gebackene Brot. Fleisch,
Milch und Brot wurden die Hauptnahrung der Menschen. Diese Art
der Sättigung durch Brot, Milch und Fleisch von Tierleichen aber ist
nicht naturgegeben. Die Organe des menschlichen Körpers sind von
Natur aus auf die Verarbeitung und Umwandlung von Gartenge­
müsen, Obst und Nüssen im Naturzustand eingerichtet und erschaffen.
Durch das Verspeisen von Brot, Milch und Fleisch samt dem gekochten
Grobgemüse müssen die Organe entarten. Dieser Zustand zeigt sich
dann in den verschiedensten Krankheitserscheinungen.

Der Boden, die Erde aber wird durch die fortschreitende Entwal­
dung, dort wo der Fleisch, Milch und Brot verzehrende Mensch den
Acker dauernd aufriß und neubestellte, das Grundwasser, das Blut
der Erde, verlieren. Die von Wald und Baumwuchs ungeschützte Erde muß unter den Strahlen der Sonne verdorren und im Laufe der Zeit zur Wüste werden. Die fruchtbare Ackererde wird in trockenen Jah­
ren entweder davonfliegen oder wegen Wassermangel versteppen und
zuletzt verdorren. Das ist bisher das Schicksal aller Kulturlandschaf­
ten gewesen, deren Bewohner sich von Ackerbau und Viehzucht zu
nähren suchten. Auch die Sahara ist durch Abholzung des einst dort
wachsenden wasserreichen Urwaldes durch menschliche Unvernunft
zur Wüste geworden.

In diesem Buche aber wird dem Menschen die natürliche ihm von
Anbeginn zugewiesene Ernährung gezeigt. Diese ist nach den Ergeb­
nissen der besten biologischen Forschungen mit allen wissenschaft­
lichen Hilfsmitteln und nach langjährigen praktischen Erfahrungen
die einzig richtige, die für den Menschen in Frage kommen kann.
Der Inhalt dieses Buches will dem Leser nicht nur die zum Ver­
ständnis notwendigen wissenschaftlichen Erkenntnisse vermitteln,
sondern ihm ganz besonders eindringlich die praktischen Erfahrungen
bringen. Nach diesen kann er sich richten, um seinen Körper in
einen Zustand vollkommener Gesundheit zu bringen. Die im folgen­
den gezeigte Art der Ernährung beruht auf den natürlichen Gesetzen
der Lebenserhaltung. Sie wird den Menschen, der sie einhalten will,
zu ungeahntem Fortschritt auf allen Gebieten seiner körperlichen,
geistigen und seelischen Fähigkeiten führen. Das aber ist die Vor­
bedingung zur Entwicklung der in den Menschen hineingeborenen
Seelenkräfte in dem gesunden Streben seines Geistes nach Voll­
kommenheit.

Die dadurch ermöglichte Gleichrichtung seiner Willensbestrebungen
mit denen der Natur wird dem Menschen den Frieden seiner Seele wie­
derfinden lassen. Dadurch wird er seine Arbeit und seine Unterneh­
mungen im Sinne der natürlichen Gegebenheiten zu gutem, friedlichem
Ende führen. Jeder folgerichtig Denkende wird durch das Lesen
dieses Buches erkennen lernen, daß alles nur erreicht werden kann
durch einfache Ernährung von natürlich gewachsener pflanzlicher
Nahrung in frischem, lebensvollem Zustand. Diese Nahrung ist nicht
nur wirtschaftlich bekömmlich und einladend, sondern zugleich wohl­
schmeckend, sättigend ·und erfrischend. Sie enthält alle Grundstoffe
für den Aufbau und den Betrieb eines gesunden Körpers.

Es muß hier gleich darauf aufmerksam gemacht werden, daß ge­
kochte Nahrung irgendeiner Art niemals natürlich sein kann, da ihre
Bestandteile durch die zerstörende Kraft der Hitzeeinwirkung bei der
Zubereitung aus ihrer lebenskräftigen, organischen Bindung heraus­
gerissen werden und dadurch für die Erhaltung der Lebenskraft des
Körpers verloren gehen. Durch das Kochen wird die Lebenskraft der
Pflanze vernichtet, die im Pflanzenwuchs gebundene Sonnenkraft und
deren Lichteinwirkung aufgelöst. Die Hitzeeinwirkung löst die auf­
bauenden mineralischen Grundstoffe der Erde aus ihrer organisch
gewachsenen Bindung und bewirkt die Bildung fester, während der
Verdauungstätigkeit nicht mehr zu lösender Verbindungen, die dann
entweder als Ballast oder als Reizgifte im Körper wirken. Schon
Temperaturerhöhungen, die über 43 Grad liegen, bringen die lebens­
spendenden Proteine (Eiweißgebilde) zum Gerinnen und töten da­
durch deren Lebenskraft. Die Stärkekörperchen im Getreide und in
den Wurzeln oder Wurzelknollen werden durch das Koch- oder Back­
verfahren gesprengt und dadurch in Kleister verwandelt. Der ent­
stehende wasserlösliche Kleister verdirbt, als Brot oder Getreide­
speise gegessen, die Verdauungsvorgänge und stört die Wandlung der
Säfte in den feinsten Haargefäßen der Blutbahnen. Er verhindert das
schnelle und störungsfreie Arbeiten der Wandlungsvorgänge in den
feinsten Muskelgewebezellen und ruft dadurch viele krankhafte
Stoffwechselstörungen mit üblen Begleiterscheinungen wie z. B. die
Zuckerharnruhr, hervor. Es ist besser, Brot und gekochte Getreide­
speisen ganz zu meiden, als sich dauernd der Gefahr des Ausbruchs
der verschiedensten Krankheitserscheinungen auszusetzen oder seinen
Körper im späteren Lebensalter verfallen zu sehen. Die organische
Zusammensetzung des natürlichen Frucht-, Trauben- und Wurzel­
zuckers in den natürlich gewachsenen Nahrungsmitteln wird in der
Hitze des Kochens vernichtet. Der Zucker wird dadurch fest und
nicht mehr so leicht wandlungsfähig wie in seinem natürlichen,
organisch gewachsenen Aufbau. Er wird deshalb für den Lebens­
betrieb so gut wie unbrauchbar. Durch den in der Siedehitze erzeug­
ten chemisch reinen Fabrikzucker entstehen Magen- und Darm­
katarrhe verschiedenster Art deshalb, weil der Kunstzucker ein
chemisch reines Erzeugnis ist. Ihm fehlt jeder natürliche Mineral­
stoffgehalt vollständig. Er wirkt deshalb wie eine fressende Säure.
In der Bratpfanne geschmolzene Öle und Fette sind, wie später
bewiesen wird, so gut wie unverdaulich und bewirken daher in den
Lebensvorgängen des Körpers schwere Störungen. Mit anderen Wor­
ten: In der Koch- und Siedehitze veränderte Nahrung wird in den
Verdauungsvorgängen nicht richtig und natürlich verarbeitet, sondern
beginnt dort zu faulen und in Gärung überzugehen. Außerdem
nimmt der weichgekochte Brei den Zähnen die Arbeit und gibt des­
halb auch keine Anregung für den Speichelfluß und die Absonderung
der Magensäfte. Er nimmt den Organen der Verdauung und Um­
wandlung die notwendige Betätigungsmöglichkeit, verwirrt, überreizt
und verdirbt den Säftefiuß der Verdauungsvorgänge und legt damit
die Grundlage zu allen Krankheitserscheinungen.

\chapter{Was ist Nahrung?}
Die einfachste Antwort auf diese Frage wäre wohl: Was wir essen,
ist unsere Nahrung. Aber der einfache Augenschein belehrt uns, wie
falsch diese Auffassung ist. Wir wissen z. B., daß ein Stück Vieh,
richtig und gesund ernährt, sich prächtig entwickelt, ein glattes Fell
zeigt und gesunde, kräftige Nachkommen zeugen kann. Wir können
demselben Tier etwas als Nahrung verabreichen, durch das es wohl
dick und fett wird, aber gleichzeitig das Fell krankhaft entartet, weil
durch sie die Drüsentätigkeit der inneren Organe und die Blutbildung
gestört wird. Daraus entstehen dann z. B. Knochenmißbildungen, die
bei Mastschweinen oft so weit getrieben werden, daß die Beine den
Körper nicht mehr tragen können. In dem einen Fall sehen wir: Was
wir dem Tier als Futter gaben, verhalf seinem Körper, seinen Organen,
seinem Fleisch, seinen Knochen und seinen Blutgefäßen usw. zur richtigen, gesunden Entwicklung. Im zweiten Beispiel wurde das Tier
durch das, was ihm gereicht wurde, krank. Wirklich gesunde Nahrung
für das Tier ist daher immer nur, was den inneren Aufbau der Organe,
der Haut, der Knochen, der Muskeln und der Gewebe richtig weiter
entwickelt, kräftigt und in bester Gesundheit erhält. Für den Menschen
die Frage „Was ist Nahrung?“ zu beantworten und die Grundgesetze
aufzustellen über das, was für ihn die richtigen und natürlichen
Nahrungsmittel sind, ist die Aufgabe dieses Buches.

Die Beantwortung ist keineswegs so einfach, wie es im ersten Augenblick
aussieht; denn wenn das, was die Menschen heute essen, für ihre
Organe, ihr Blut und ihre Knochen, für die Entwicklung ihrer inneren
Säfte usw. das Richtige wäre, dann müßten sich alle Menschen in
prächtiger Gesundheit entwickeln. Sie müßten alle, ohne Ausnahme, ein
hohes und zufriedenes Alter erreichen und bis zum Schluß schaffensfreudig,
kräftig und leistungsfähig sein. Sie würden dann nicht, wie es
heute an der Tagesordnung ist, vorzeitig und mit Schmerzen sterben,
sondern einem normalen, zufriedenen Alterstod erliegen. Sie würden
dann nicht einem Versagen der inneren Organe zum Opfer fallen oder
an Krankheiten körperlicher, geistiger oder seelischer Art leiden, durch
die sie massenweise in Krankenhäusern, Irrenanstalten und Gefängnissen verderben.
Was ist denn die natürliche Nahrung des Menschen?
Ehe wir diese Frage beantworten können, müssen wir uns ein wenig
in der Natur umsehen, um zu erfassen, aus was eigentlich das Leben
besteht, das wir durch unsere Nahrung aufrecht erhalten wollen.

\input{1/1/2}



\newpage

%=========================================
%\blinddocument


%=========================================
\begin{comment}
Just some notes, not visible in pdf.
\end{comment}


\end{document}
